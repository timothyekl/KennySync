\documentclass{article}

\usepackage{fancyhdr}
\usepackage{mdwlist}

\pagestyle{fancy}
\headheight 35pt
\headsep 16pt
\begin{document}

\lhead{\textbf{Project Proposal} \\ CSSE 433}
\rhead{Will Anderson \\ Tim Ekl \\ Eric Reed}

\section{Executive Summary}

This document briefly describes KennySync, an implementation of the Paxos conflict-resolution algorithm. KennySync is prepared as a project for the Advanced Databases course at Rose-Hulman Institute of Technology by Will Anderson, Tim Ekl, and Eric Reed. Implemented in Ruby, KennySync is aimed to be a complete, standalone project, with the option of expanding to function as a wrapper around a distributed database system (such as Redis).

\section{KennySync: A Paxos Implementation}

With the rise of concurrent processing and massively scalable systems, consistency and accuracy are of more concern than ever in data storage and access. KennySync is a project to implement the Paxos algorithm, introduced by Leslie Lamport in 1998, in order to provide stable, consistent data storage and retrieval.

\subsection{Project Goals}

Due to the uncertain nature of integrating with some other systems under consideration, the goals of the KennySync project are separated into two major categories: those which are mandatory for project completion, and those which are contingent on time available. By its completion, KennySync \textit{must}:

\begin{itemize*}
\item Be runnable as a node in a Paxos implementation
\item Function as all roles described in ``basic'' Paxos, including proposer, accepter, and learner
\item Propagate values through the network of nodes, and allow those values to be retrieved
\item Provide some visualization into the system, for verifying functionality and watching value propoagation
\item Be fully failure-resistant and scalable as described by Lamport
\end{itemize*}

KennySync also \textit{may}:

\begin{itemize*}
\item Integrate or wrap an existing database system, such as Redis or SQLite
\item Collect and provide statistics about the network of nodes
\end{itemize*}

\subsection{Implementation}


\end{document}
