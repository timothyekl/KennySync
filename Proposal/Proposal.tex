\documentclass{article}

\usepackage{fancyhdr}
\usepackage[colorlinks=true]{hyperref}
\usepackage{mdwlist}

\pagestyle{fancy}
\headheight 35pt
\headsep 16pt
\begin{document}

\lhead{\textbf{Project Proposal} \\ CSSE 433}
\rhead{Will Anderson \\ Tim Ekl \\ Eric Reed}

\section{Executive Summary}

This document briefly describes KennySync, an implementation of the Paxos conflict-resolution algorithm. KennySync is prepared as a project for the Advanced Databases course at Rose-Hulman Institute of Technology by Will Anderson, Tim Ekl, and Eric Reed. Implemented in Ruby, KennySync is aimed to be a complete, standalone project, with the option of expanding to function as a wrapper around a distributed database system (such as Redis).

\section{KennySync: A Paxos Implementation}

With the rise of concurrent processing and massively scalable systems, consistency and accuracy are of more concern than ever in data storage and access. KennySync is a project to implement the Paxos algorithm, introduced by Leslie Lamport in 1998, in order to provide stable, consistent data storage and retrieval.

\subsection{Project Goals}

Due to the uncertain nature of integrating with some other systems under consideration, the goals of the KennySync project are separated into two major categories: those which are mandatory for project completion, and those which are contingent on time available. By its completion, KennySync \textit{must}:

\begin{itemize*}
\item Be runnable as a node in a Paxos implementation
\item Function as all roles described in ``basic'' Paxos, including proposer, accepter, and learner
\item Propagate values through the network of nodes, and allow those values to be retrieved
\item Provide some visualization into the system, for verifying functionality and watching value propagation
\item Be fully failure-resistant and scalable as described by Lamport
\end{itemize*}

KennySync also \textit{may}:

\begin{itemize*}
\item Integrate or wrap an existing database system, such as Redis or SQLite
\item Collect and provide statistics about the network of nodes
\end{itemize*}

\subsection{Implementation}

KennySync will be implemented in the Ruby 1.9 programming language, and is aimed at cross-platform compatibility between the Mac OS X and Ubuntu Linux platforms. Development will use GitHub as a central location for storage and communication of code, tickets, and documentation. Due to the distributed nature of GitHub and the size of the team, KennySync will use an iterative development process, with weekly status meetings between team members and individual development sprints.

\section{The Team}

The members of the KennySync team are:

\begin{itemize*}
\item Will Anderson, senior student in Computer Science
\item Tim Ekl, graduate student in Engineering Management
\item Eric Reed, senior student in Computer Science and Mathematics
\end{itemize*}

\section{Literature Review}

To begin work and gain a conceptual understanding of both the Paxos algorithm and current work in the field, the team intends to read the following papers describing variations on the Paxos algorithm (all by Leslie Lamport):

\begin{itemize*}
\item A simple description of Paxos \cite{simple-paxos}
\item The fuller, original paper describing Paxos \cite{paxos}
\item A version of Paxos that trades fault tolerance for cheap computations \cite{cheap-paxos}
\item A version of Paxos that runs faster \cite{fast-paxos}
\item A generalization of the ``fast Paxos'' above \cite{generalized-paxos}
\end{itemize*}

\bibliographystyle{plain}
\bibliography{Proposal}

\end{document}
